\documentclass[titlepage]{article}
\usepackage{babel}
\usepackage{amsmath}
\usepackage{amssymb}
\usepackage{amsthm}
\usepackage{multicol} %spalten in seite
\usepackage{graphicx} %bilder einfügen
\usepackage{tabto} %tabulator mit \tab
\usepackage{hyperref}
\usepackage[T1]{fontenc}
\usepackage{mathrsfs}  

\usepackage[utf8]{inputenc}
\usepackage{listings} %quellcode
\pagenumbering{arabic}
\renewcommand{\arraystretch}{1.3} %vertikaler abstand von tabellen
\newcommand{\n}{\newline}
\newcommand{\A}{\mathbb{A}}
\newcommand{\B}{\mathbb{B}}
\newcommand{\C}{\mathbb{C}}
\newcommand{\D}{\mathbb{D}}
\newcommand{\E}{\mathbb{E}}
\newcommand{\F}{\mathbb{F}}
\newcommand{\G}{\mathbb{G}}
\renewcommand{\H}{\mathbb{H}}
\newcommand{\I}{\mathbb{I}}
\newcommand{\J}{\mathbb{J}}
\newcommand{\K}{\mathbb{K}}
\renewcommand{\L}{\mathbb{L}}
\newcommand{\M}{\mathbb{M}}
\newcommand{\N}{\mathbb{N}}
\renewcommand{\O}{\mathbb{O}}
\renewcommand{\P}{\mathbb{P}}
\newcommand{\Q}{\mathbb{Q}}
\newcommand{\R}{\mathbb{R}}
\renewcommand{\S}{\mathbb{S}}
\newcommand{\T}{\mathbb{T}}
\newcommand{\U}{\mathbb{U}}
\newcommand{\V}{\mathbb{V}}
\newcommand{\W}{\mathbb{W}}
\newcommand{\X}{\mathbb{X}}
\newcommand{\Y}{\mathbb{Y}}
\newcommand{\Z}{\mathbb{Z}}

\usepackage[left=20mm, right=15mm, top=25mm, bottom=30mm, paper=a4paper]{geometry}
\pagestyle{plain}


\begin{document}
	
	\title{Diskrete Strukturen - Übung 04}
	\author{Felix Tischler, Martrikelnummer: 191498}
	\date{\today}
	\maketitle
	
	\part*{Fibonacci-Zahlen}
	\section*{1.)}Beweisen Sie durch vollständige Induktion über $m$, dass das sogenannte "Additionstheorem" für Fibonacci-Zahlen für alle $n\ge1$ und $m\ge2$ gilt:
		\begin{align*}
			f_{m+n}&=f_{m+1}\cdot f_n+f_m\cdot f_{n-1}
		\end{align*}
		\subsection*{Induktionsanfang} falls $m=1$ und falls $m=2$:\\
			\scalebox{1.2}{\parbox{.5\linewidth}{%
			\begin{align*}
				f_{n+1}&=f_n+f_{n-1}=f_2\cdot f_n+f_1\cdot f_{n-1}=1\cdot f_n+1\cdot f_{n-1}\\
				f_{2+n}&=f_{n+2}=f_{n+1}+f_n=f_n+f_{n-1}+f_n=2\cdot f_n+1\cdot f_{n-1}\\
				&=f_{3}\cdot f_n+f_2\cdot f_{n-1}\\
			\end{align*}
			}}
		\subsection*{Induktionsvoraussetzung IV}
		mit $m=k, n\in\mathbb{N}$ und $n\ge1,m\ge2$:\\
			\scalebox{1.2}{\parbox{.5\linewidth}{%
					\begin{align*}
						f_{n}&=f_{n-1}+f_{n-2}\\
						f_{n+k}&=f_{k+1}\cdot f_n+f_k\cdot f_{n-1}\\
						f_{n+k-1}&=f_k\cdot f_n+f_{k-1}\cdot f_{n-1}\\
					\end{align*}
			}}
		\subsection*{Induktionsbehauptung}
		mit $m=k+1$:\\
			\scalebox{1.2}{\parbox{.5\linewidth}{%
					\begin{align*}
						f_{k+1+n}&=f_{k+2}\cdot f_n+f_{k+1}\cdot f_{n-1}
					\end{align*}
			}}
		\subsection*{Induktionsbeweis}
			\scalebox{1.2}{\parbox{.5\linewidth}{%
					\begin{align*}
						f_{k+1+n}&=f_{n+(k+1)}=f_{n+k}+f_{n+k-1} \mid\text{ mit IV}\\
						&\Rightarrow f_{k+1}\cdot f_n+f_k\cdot f_{n-1}+f_{n+k-1} \mid\text{mit IV}\\
						&\Rightarrow f_{k+1}\cdot f_n+f_k\cdot f_{n-1}+f_k\cdot f_n+f_{k-1}\cdot f_{n-1}\\
						&=(f_{k+1}+f_k)\cdot f_n+(f_k+f_{k-1})\cdot f_{n-1}\\
						&=(f_{k+2})\cdot f_n+(f_{k+1})\cdot f_{n-1} \qed
					\end{align*}
			}}
		
	\section*{2.)}Nun betrachten wir folgende Gleichung für die Fibonacci-Zahlen:
		\begin{align*}
			\sum^n_{i=1}f^2_i=f_n\cdot f_{n+1}
		\end{align*}
		\subsection*{a)}Beweisen Sie diese Gleichung durch vollständige Induktion über n.
			\subsection*{Induktionsanfang} falls: $n=1$\\
			\scalebox{1.2}{\parbox{.5\linewidth}{%
					\begin{align*}
						f^2_1&=f_1\cdot f_{2}\\
						1^2&=1\cdot1
					\end{align*}
			}}
			\subsection*{Induktionsvoraussetzung IV}
			mit $n=k\in\mathbb{N}$:\\
			\scalebox{1.2}{\parbox{.5\linewidth}{%
					\begin{align*}
						\sum^k_{i=1}f^2_i=f_k\cdot f_{k+1}
					\end{align*}
			}}
			\subsection*{Induktionsbehauptung}
			mit $n=k+1$:\\
			\scalebox{1.2}{\parbox{.5\linewidth}{%
					\begin{align*}
						\sum^{k+1}_{i=1}f^2_i=f_{k+1}\cdot f_{k+2}
					\end{align*}
			}}
			\subsection*{Induktionsbeweis}
			\scalebox{1.2}{\parbox{.5\linewidth}{%
					\begin{align*}
						\sum^{k+1}_{i=1}f^2_i&=\sum^k_{i=1}f^2_i+f^2_{k+1} \mid\text{mit IV}\\
						&\Rightarrow f_k\cdot f_{k+1}+f_{k+1}\cdot f_{k+1}\\
						&=f_{k+1}\cdot(f_k+f_{k+1})\\
						&=f_{k+1}\cdot f_{k+2} &\qed
					\end{align*}
			}}
		\subsection*{b)}Versuchen Sie einen alternativen Beweis, der ohne Induktion auskommt, indem bereits bewiesene Gleichungen für Fibonacci-Zahlen verwendet werden.
			\begin{align*}
				f_n\cdot f_{n+1} &=(f_{n-1}+f_{n-2})\cdot(f_n+f_{n-1})\\
				&=f_{n-1}\cdot f_n+f^2_{n-1}+f_{n-2}\cdot f_n+f_{n-2}\cdot f_{n-1}\\
				&=f^2_{n-1}+f_n\cdot(f_{n-1}+f_{n-2})+f_{n-2}\cdot f_{n-1}\\
				&=\sum^{n-2}_{i=1}f^2_i+f^2_{n-1}+f^2_n=\sum^n_{i=1}f^2_i&\qed
			\end{align*}
	\section*{3.)} Zum Abschluss formulieren Sie einen Beweiß für eine weitere Identität der Fibonacci-Zahlen:
		\begin{align*}
			\sum^n_{k=1}f_{3k}=\frac{1}{2}\cdot(f_{3n+2}-1)
		\end{align*}
					\subsection*{Induktionsanfang} falls $n=1$:\\
		\scalebox{1.2}{\parbox{.5\linewidth}{%
				\begin{align*}
					\frac{1}{2}\cdot(f_5-1)=\frac{1}{2}\cdot4=2=f_3
				\end{align*}
		}}
		\subsection*{Induktionsvoraussetzung IV}
		mit: $n=k,i\in\mathbb{N}$\\
		\scalebox{1.2}{\parbox{.5\linewidth}{%
				\begin{align*}
					\sum^k_{i=1}f_{3i}=\frac{1}{2}\cdot(f_{3k+2}-1)
				\end{align*}
		}}
		\subsection*{Induktionsbehauptung}
		mit: $n=k+1$\\
		\scalebox{1.2}{\parbox{.5\linewidth}{%
				\begin{align*}
					\sum^{k+1}_{i=1}f_{3i}=\frac{1}{2}\cdot(f_{3(k+1)+2}-1)=\frac{1}{2}\cdot(f_{3k+5}-1)
				\end{align*}
		}}
		\subsection*{Induktionsbeweis}
		\scalebox{1.2}{\parbox{.5\linewidth}{%
				\begin{align*}
					\sum^{k+1}_{i=1}f_{3i}&=\sum^k_{i=1}f_{3i}+f_{3(k+1)}\mid\text{mit IV}\\
					&\Rightarrow \frac{1}{2}\cdot(f_{3k+2}-1)+f_{3k+3}\\
					&=\frac{1}{2}\cdot(f_{3k+2}+2f_{3k+3}-1)\\
					&=\frac{1}{2}\cdot(f_{3k+2}+f_{3k+3}+f_{3k+3}-1)\\
					&=\frac{1}{2}\cdot(f_{3k+4}+f_{3k+3}-1)\\
					&=\frac{1}{2}\cdot(f_{3k+5}-1) &\qed
				\end{align*}
		}}
\end{document}
